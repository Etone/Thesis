%Glossar
\newglossaryentry{gls-k8}{name={Kubernetes},description={Kubernetes ist ein Open-Source Container Orchestrierungssystem}}
\newglossaryentry{gls-eponym}{name={Eponym},description={Ein Eponym ist ein Eigenname, der sich als Markenname durchgesetzt hat. Bsp. googlen, Tempo-Taschentuch oder Tesa}}
\newglossaryentry{gls-eureka}{name={Eureka},description={Eurka ist eine von Netflix OSS entwickelte Software zur Service Discovery. Eureka wird im Rahmen der Spring Cloud Services entwickelt und veröffentlicht}}
\newglossaryentry{gls-devops}{name={DevOps},description={DevOps (von \textit{Development} und \textit{Operations}) dienen der einfacheren Auslieferung von Software an Entwickler wie an den Kunden. Dabei verwenden sie \gls{acr-ci} Tools wie Jenkins um eine automatisierte Bereitstellung zu gewährleitsen}}
%Akronyme
\newacronym{acr-cgroup}{cgroup}{control group}
\newacronym{acr-oci}{OCI}{Open Container Initiative}
\newacronym{acr-k8}{K8}{Kubernetes}
\newacronym{acr-cncf}{CNCF}{Cloud Native Computing Foundation}
\newacronym{acr-cf}{CF}{Cloud Foundry}
\newacronym{acr-aws}{AWS}{Amazon Web Services}
\newacronym[
plural=VMs,
firstplural=Virtuelle Machinen (VMs)
]{acr-vm}{VM}{Virtuelle Machine}
\newacronym{acr-ci}{CI}{Continous Integration}
\newacronym{acr-os}{OS}{Operating System}