%Glossar
\newglossaryentry{gls-devops}{name={DevOps},description={DevOps (von \textit{Development} und \textit{Operations}) dienen der einfacheren Auslieferung von Software an Entwickler wie an den Kunden. Dabei verwenden sie \gls{acr-ci} Tools wie Jenkins um eine automatisierte Bereitstellung zu gewährleitsen}}
\newglossaryentry{gls-image}{name={Image},description={Ein Container Image ist eine Datei, die spezifiziert, wie ein Container vond er Laufzeitumgebung ausgeführt werden soll}}
\newglossaryentry{gls-cn}{name={Cloud-Native},description={Cloud-Native Anwendungen sind spezifisch für Cloud-Architekturen entwickelt. Sie spalten meistens große Funktionalitäten in kleine Microservices, die mittels API miteinander kommunizieren und sind somit skalierbar, loose gekoppelt und ausfallsicherer als Full-CLient Anwendungen}}
\newglossaryentry{gls-bash}{name={Bash},description={Bash ist eine freie, umfangreiche Shell, die in den meisten Unix-Systemem Standard ist}}


%Akronyme
\newacronym{acr-cgroup}{cgroup}{control group}
\newacronym{acr-oci}{OCI}{Open Container Initiative}
\newacronym{acr-k8}{K8}{Kubernetes}
\newacronym{acr-cncf}{CNCF}{Cloud Native Computing Foundation}
\newacronym{acr-cf}{CF}{Cloud Foundry}
\newacronym{acr-aws}{AWS}{Amazon Web Services}
\newacronym[
plural=VMs,
firstplural=Virtuelle Machinen (VMs)
]{acr-vm}{VM}{Virtuelle Machine}
\newacronym{acr-ci}{CI}{Continous Integration}
\newacronym{acr-os}{OS}{Betriebssystem (\emph{Operating System})}
\newacronym{acr-rkt}{rkt}{Rocket}
\newacronym{acr-chroot}{chroot}{Change Root}
\newacronym{acr-appc}{appc}{App Container}
\newacronym{acr-saas}{SaaS}{Software as a Service}
\newacronym{acr-veth}{VEth}{virtual Ethernet}
\newacronym{acr-cli}{CLI}{Command Line Interface}
\newacronym{acr-pid}{PID}{Process Identifier}