\documentclass[
fontsize=11pt,
liststotoc,
idxtotocnumbered,
parskip,
oneside,
DIVcalc,
table,
]{scrbook}

%Using Directives
%Deutsche Sprache und Buchstaben
\usepackage[ngerman]{babel}
\usepackage[utf8]{inputenc}

%Bilder, Grafiken
\usepackage{graphicx}

%Tabellen
\usepackage{booktabs}
\usepackage{tabu}
\renewcommand{\arraystretch}{1.3}

%Schönere Code Listings
\usepackage{minted}

%Frame um gewünschte Umgebung
\usepackage{mdframed}
\surroundwithmdframed{minted}

%Formeln
\usepackage{amsmath}

%Literaturverzeichnis
\usepackage[
backend=biber,			%Biber statt Biblatex, da biber zuverlässiger funktioniert
style=iso-authoryear,	%ISO 690 Style, (Author, Jahr)
natbib					%befehle citep und citet für klammern um Quellenangabe
]{biblatex}
\bibliography{literatur}

%Glossar
\usepackage[
toc,					%Eintrag im Inhaltsverzeichnis
acronym					%Acronyme mit drucken
]{glossaries}

%Nicht indiziertes Inhaltsverzeichnis, sortiert durch LaTeX, keine Umlaute in Kürzeln, weil sonst Probleme
\makenoidxglossaries
%laden der glossar Datei für alle Einträge
\loadglsentries{glossar}

%Header und Footer Formatierung
\usepackage{fancyhdr}

\renewcommand{\chaptermark}[1]{\markboth{\MakeUppercase{\thechapter.\ #1}}{}}
%redefiniert plain style für Chapter seiten (Erste seite des Chapters)
\fancypagestyle{plain}
{
	\fancyhf{}
	\fancyfoot[R]{\thepage}
	\fancyfoot[L]{\thetitle}
	\renewcommand{\headrulewidth}{0pt}
	\renewcommand{\footrulewidth}{0.4pt}
}

%allgemein genutzer style, beschriftung des Chapters
\fancypagestyle{main}{
	\fancyhf{}
	\fancyfoot[L]{\thetitle}
	\fancyfoot[R]{\thepage}
	\fancyhead[L]{\MakeUppercase\leftmark}
	\renewcommand{\headrulewidth}{0.4pt}
	\renewcommand{\footrulewidth}{0.4pt}
}

%Genutzter stil für ehernwörtliche Erklärung, ToC und Zusammenfassung /Sperrvermerk
%wenn länger als eine Seite, sonst plain
\fancypagestyle{front}{
	\pagenumbering{Roman}
	\fancyhf{}
	\fancyfoot[R]{\thepage}
	\fancyfoot[L]{\thetitle}
	\renewcommand{\headrulewidth}{0pt}
	\renewcommand{\footrulewidth}{0.4pt}	
}

%Genutzt für Anhang, keine Kopfleiste, alphabetische Seitenzahlen
\fancypagestyle{back}{
	\pagenumbering{Alph}
	\fancyhf{}
	\fancyfoot[R]{\thepage}
	\fancyfoot[L]{\thetitle}
	\fancyhead[L]{\leftmark}
	\renewcommand{\headrulewidth}{0.4pt}
	\renewcommand{\footrulewidth}{0.4pt}	
}

%befehle \thisauthor und \thistitle um Title und Autor nur einmalig zu setzen und nicht immer wieder, wichtig für Fußzeile
\usepackage{titling}

%Fancy Referenzen
\usepackage[
german,		%Deutsche Begriffe (z.B. Abblidung statt Figure)
plain		%Sytle der Referenz, normaler Text
 ]{fancyref}
 
%Hinzufügen von Listings mit lst
\newcommand*{\fancyreflstlabelprefix}{lst}
\fancyrefaddcaptions{german}{%
	\newcommand*{\Freflstname}{Listing}%
	\newcommand*{\freflstname}{\Freflstname}%
}
\frefformat{plain}{\fancyreflstlabelprefix}{%
	\freflstname\fancyrefdefaultspacing#1%
}


%Färben von einzelnen Stellen (genutzt für Tabellen)
\usepackage[
x11names		%Mehr Farben
]{xcolor}

%Durchgehende Beschriftungen von z.B. Abbildungen
\usepackage{chngcntr}

%meherer Abbildungen nebeneinander
\usepackage{subfigure}

%Font Awesome Icons
\usepackage{fontawesome}

%Directory tree
\usepackage{dirtree}

%Link zu seiten usw.
\usepackage[
hidelinks		%Keine roten Boxen um Links
]{hyperref}

%Todos
\usepackage{todonotes}

%Commands
\newcommand{\tildeawesome}{{\raise.17ex\hbox{$\sim$}}}