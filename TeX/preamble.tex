% !TeX spellcheck = de_De
\documentclass[
fontsize=11pt,
bibtoto,				%Literaturverzeichnis in Inhaltsverzeichnis aufnehmen
liststotoc,
idxtotocnumbered,
parskip,
oneside,
DIVcalc,
table
]{scrbook}

%Using Directives
%Deutsche Sprache und Buchstaben
\usepackage[ngerman]{babel}
\usepackage[utf8]{inputenc}

%Bilder, Grafiken
\usepackage{graphicx}

%Tabellen
\usepackage{booktabs}

%Codelistings
\usepackage{listingsutf8}
\lstset{inputencoding=utf8/latin1,
	captionpos=b, 		%Beschriftung unter Listing
	frame=single,		%Rahmen um Listing
}

%Formeln
\usepackage{amsmath}

%Literaturverzeichnis
\usepackage[
backend=biber,			%Biber statt Biblatex, da biber zuverlässiger funktioniert
style=iso-authoryear,	%ISO 690 Style, (Author, Jahr)
natbib					%befehle citep und citet für klammern um Quellenangabe
]{biblatex}
\bibliography{literatur}

%Glossar
\usepackage[
toc,					%Eintrag im Inhaltsverzeichnis
acronym					%Acronyme mit drucken
]{glossaries}

%Nicht indiziertes Inhaltsverzeichnis, sortiert durch LaTeX, keine Umlaute in Kürzeln, weil sonst Probleme
\makenoidxglossaries
%laden der glossar Datei für alle Einträge
\loadglsentries{glossar}

%Header und Footer Formatierung
\usepackage{fancyhdr}
%redefiniert plain style für Chapter seiten (Erste seite des Chapters)
\fancypagestyle{plain}
{
	\fancyhf{}
	\fancyfoot[R]{\thepage}
	\fancyfoot[L]{\thetitle}
	\renewcommand{\headrulewidth}{0pt}
	\renewcommand{\footrulewidth}{0.4pt}
}

\renewcommand{\chaptermark}[1]{\markboth{#1}{}}

%allgemein genutzer style, beschriftung des Chapters
\fancyhf{}
\fancyfoot[L]{\thetitle}
\fancyhead[L]{\leftmark}
\fancyfoot[R]{\thepage}
\renewcommand{\headrulewidth}{0.4pt}
\renewcommand{\footrulewidth}{0.4pt}
%Alle Pagestyles zu fancy setzen, ändert Kopf und Fußzeilen
\pagestyle{fancy}

%befehle \thisauthor und \thistitle um Title und Autor nur einmalig zu setzen und nicht immer wieder, wichtig für Fußzeile
\usepackage{titling}

%Tikz ist kein Zeichenprogramm
\usepackage{tikz}

%Fancy Referenzen
\usepackage[
german,		%Deutsche Begriffe (z.B. Abblidung statt Figure)
plain		%Sytle der Referenz, normaler Text
 ]{fancyref}

%Link zu seiten usw.
\usepackage[
hidelinks		%Keine roten Boxen um Links
]{hyperref}

%Färben von einzelnen Stellen (genutzt für Tabellen)
\usepackage{xcolor}

%Testtext
\usepackage{blindtext}
