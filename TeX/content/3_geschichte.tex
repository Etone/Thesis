\chapter{Geschichte}
\label{chap:geschichte}
Um die Frage zu beantworten, wie Docker die führende Container-Technologie wurde, wird im Folgenden die Geschichte dieser betrachtet. Dabei liegt der Schwerpunkt auf Lösungen und Probleme der Technologien.
\section{Zeitlicher Überblick}
\label{sec:timeline}
In \fref{tab:timelineContainers} wird ein Überblick über die Geschichte der Container-Isolation gegeben. Diese zeigt Schritte in der Technologie, Entwickelte Container-Runtimes und aktuelle Themen wie die Orchestrierung von Containern.
\begin{table}[h]
	\newcommand{\style}{\color{LightSteelBlue3}\makebox[0pt]{\textbullet}\hskip-0.5pt\vrule width 1pt\hspace{\labelsep}}
	\begin{center}
		\begin{tabular}{@{\,}r <{\hskip 2pt} !{\style} >{\raggedright\arraybackslash}p{7cm}}
			\addlinespace[1.5ex]
			\toprule
			\multicolumn{2}{c}{\textbf{Benötigte Technologien}}		\\
			1979 & Unix V7 											\\
			2000 & FreeBSD Jails									\\
			2001 & Linux VServer									\\
			2004 & Solaris Container								\\
			2005 & Open VZ											\\
			2006 & Google Process Container							\\
			\midrule
			\multicolumn{2}{c}{\textbf{Container Runtimes}}			\\
			2008 & LXC												\\
			2011 & CF Warden										\\
			2013 & \gls{acr-lmctfy}									\\
			2013 & Docker											\\
			2014 & rkt												\\
			2015 & LXD												\\
			\midrule
			\multicolumn{2}{c}{\textbf{Orchestrierung und andere Themen}}\\
			2014 & Entwicklung \gls{acr-k8} startet					\\
			2016 & Container Security durch dirty COW				\\
			2017 & Übernahme rkt und containerd in \gls{acr-cncf}	\\
			
			\bottomrule
		\end{tabular}
		\caption{Timeline der wichtigsten Container-Technologien \citep{ABriefHistoryofContainers:fromthe1970sto2017}}
		\label{tab:timelineContainers}
	\end{center}
\end{table}