\chapter{Geschichte}
\label{chap:geschichte}
Um die Frage zu beantworten, wie Docker die führende Container-Technologie wurde, wird im Folgenden die Geschichte dieser betrachtet. Dabei liegt der Schwerpunkt auf Lösungen und Probleme der Technologien.
\section{Zeitlicher Überblick}
\label{sec:timeline}
 
In \fref{tab:timelineContainers} wird ein Überblick über die Geschichte der Container-Isolation gegeben. Diese zeigt Schritte in der Technologie, Entwickelte Container-Runtimes und aktuelle Themen wie die Entwicklung hinter Container-Ecosystems.
 
\begin{table}
	\newcommand{\timeline}{\color{LightSteelBlue3}\makebox[0pt]{\textbullet}\hskip-0.5pt\vrule width 1pt\hspace{\labelsep}}
	\begin{center}
		\begin{tabular}{@{\,}r <{\hskip 3pt} !{\timeline} >{\raggedright\arraybackslash}p{9cm}}
			\toprule
			\multicolumn{2}{c}{\textbf{Benötigte Technologien}}		\\
			\midrule
			1979 & Unix V7 mit chroot								\\
			1998 & SELinux											\\
				 & AppArmor											\\
			1999 & Linux Capabilites								\\
			2000 & FreeBSD Jails									\\
			2001 & Linux VServer									\\
			2002 & Linux namespaces									\\
			2004 & Solaris Container								\\
			2005 & Open VZ											\\
			2006 & Google Process Container							\\
			2007 & Process Container in Linux Kernel als \glspl{acr-cgroup}\\
			2012 & Erste Stabile Version der Sprache Go					\\
			\midrule
			\multicolumn{2}{c}{\textbf{Container-Runtimes}}			\\
			\midrule
			2008 & \gls{acr-lxc}									\\
			2011 & \gls{acr-cf} Warden								\\
			2013 & \gls{acr-lmctfy}									\\
				 & \gls{acr-cf} Graden, Umstieg auf Go				\\
			2014 & \Gls{acr-appc} Spezifikation Release				\\
				 & rkt												\\
			2015 & LXD												\\
				 & runC												\\
			2016 & Windows Containers								\\
				 & \gls{acr-cf} Guardian Release, Support für runC	\\
			2017 & containerd v1.0.0								\\
				 & cri-o											\\
			\midrule
			\multicolumn{2}{c}{\textbf{Entwicklung Container-Ecosystem}}\\
			\midrule
			2011 & Initialer Release \gls{acr-cf}					\\
			2013 & Release Docker, erstes Container Ecosystem		\\
			2014 & Entwicklung \gls{acr-k8} startet					\\
			2015 & Gründung \gls{acr-cncf} und \gls{acr-oci}		\\
				 & Docker Swarm										\\
			2016 & "Dirty Cow" $\rightarrow$ Container Sicherheit	\\
				 & Apache Mesos v1.0.0 Release						\\
			2017 & Übernahme rkt und containerd in \gls{acr-cncf}	\\
				 & Release \gls{acr-oci} runtime-spec und image-spec\\
			\bottomrule
		\end{tabular}
	\caption{Timeline Container-Technologien \citep{ABriefHistoryofContainers:fromthe1970sto2017}}
	\label{tab:timelineContainers}
	\end{center}
\end{table}

\section{Container-Engines: Von LXC zu Docker}
\label{sec:lxc2containerd}
Wie in \fref{tab:timelineContainers} zu sehen, sind Container keine neue Erfindung. Bereits 2008 wurde die erste volle Implementierung einer Container-Runtime mit \gls{acr-lxc} veröffentlicht. Im folgenden Abschnitt wird ein Blick auf Container-Runtimes in der Vergangenheit geworfen und die Frage geklärt, wie Docker so erfolgreich wurde. Dazu werden Probleme älterer Engines anhand eines Beispiels aufgezeigt und Lösungen, die jüngere Schritte mit sich bringen, erklärt.
\todo{Welches Beispiel?}

\todo{Basically Everything, das Chapter schreiben}

\subsection{Vor LXC: Isolation mit Kernelpatches und Virtualisierung}
\label{sec:geschichteVorLXC}

\subsection{LXC: Erste Schritte in Container-Runtimes}
\label{sec:geschichteLXC}

\subsection{CF Warden: Innovation durch Vereinfachung}
\label{sec:geschichteCFWarden}

\subsection{Go: Bedeutung der Sprache für Container}
\label{sec:geschichteGo}
\todo{Exkursion? Oder Streichen}

\subsection{LMCTFY: Open-Source und Kooperation}
\label{sec:geschichteLMCTFY}

\subsection{Docker: Ecosystem, Runtime und \gls{acr-saas} in einem}
\label{sec:geschichteDocker}

\section{Aktuelle Probleme und Lösungen: Orchestrierung, Sicherheit und Standards}
\label{sec:geschichteAktuell}