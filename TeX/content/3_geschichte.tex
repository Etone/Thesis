\chapter{Geschichte}
\label{chap:geschichte}
Um die Frage zu beantworten, wie Docker die führende Container-Technologie wurde, wird im Folgenden die Geschichte dieser betrachtet. Dabei liegt der Schwerpunkt auf Lösungen und Probleme der Technologien.
 
\section{Container-Engines: Von LXC zu Docker}
\label{sec:lxc2containerd}
Wie in \fref{tab:timelineContainers} zu sehen, sind Container keine neue Erfindung. Bereits 2008 wurde die erste volle Implementierung einer Container-Runtime mit LXC ver"-öffentlicht. Im folgenden Abschnitt wird ein Blick auf Container-Runtimes in der Vergangenheit geworfen und die Frage geklärt, wie Docker so erfolgreich wurde. Dazu werden Probleme älterer Container-Engines und Lösungen, die jüngere Schritte mit sich bringen, erklärt.

\subsection{Vor LXC: Isolation mit Kernel-Patches}
\label{sec:geschichteVorLXC}

Bereits 1979 wurde mit \gls{acr-chroot} die erste, noch heute notwendige, Funktion veröffentlicht. Diese kam mit dem Betriebssystem Unix V7 und erlaubte es erstmals, verschiedene Prozesse in unterschiedliche Dateisystemen zu trennen. Der Systemaufruf wurde 1982 in BSD hinzugefügt \citep{ABriefHistoryofContainers:fromthe1970sto2017}. 20 Jahre später rücken isolierbare Prozesse durch FreeBSD Jails in den Mittelpunkt. Diese erweitern das \Gls{acr-chroot}-Konzept, indem nicht nur das Dateisystem vom Host-System getrennt wird, sondern auch Hostnamen, IP-Adressen und die Nutzerverwaltung \citep{FreeBSDHandbook}. 

Neben FreeBSD Jails wurde bereits 2001 Linux VServer veröffentlicht, eine Software, die ähnlich wie Jails eine Isolation des Dateisystems und auch der Netzwerkadresse erlaubt. Entgegen der aktuell noch weiterentwickelten Jails war der letzte stabile Release von VServer 2008 \citep{PaperLinuxVServer}. Der größte Nachteil des VServers waren die bnötigten Kernel-Patches, die benötigt wurden, um die Isolierung zu gewährleisten.

In den folgenden Jahren wurden immer mehr Lösungen zur Isolation von Prozessen veröffentlicht, darunter Oracles Solaris Containers, das auf Zonen im Betriebssystem setzt und Open VZ, welches wie VServer, einen gepatchten Linux-Kernel benötigt, aber auch Ressourcen isolieren kann. Der größte Nachteil, denn alle diese Technologien haben, ist die unzureichende und komplizierte Virtualisierung einzelner Prozesse. Zudem muss man Funktionen, die zur Isolation benötigt werden, nachpatchen. Dies führte 2006 dazu, dass Entwickler von Google eine bessere Lösung entwickelten, Process Containers. Diese erlaubten ohne Patches eine einfache Verwendung und Isolation einzelner Ressourcen. Im Jahr 2007 wurden Process Container unter dem Namen \glspl{acr-cgroup} in den Linux-Kernel gemerged und liefern seitdem das Fundament für aktuelle Container-Technologien.

\subsection{LXC: Erste Schritte in Container-Runtimes}
\label{sec:geschichteLXC}
Mit \gls{acr-lxc} kam 2008 die erste vollwertige Implementation, die alleine mit dem nativen Kernel des Linux-\gls{acr-os} funktioniert. Damit löst LXC eins der größten Probleme der vorherigen Lösungen, indem kein gepachter Kernel benötigt wird.
\begin{table}[h]
		\begin{tabular}{ll}
			\toprule
			Feature & Verwendung\\
			\midrule
			Namespaces & Trennung unterschiedlicher Systemkomponenten zur Virtualisierung\\
			Apparmor und SELinux & Mandatory Access Control, Verwaltung\\
			Seccomp & Sicherheitsprofile, sperrt Systemaufrufe des isolierten Prozesses\\
			chroot und pivot\_root & Isolation des Dateisystems und Nutzernamespace-Mapping\\
			Capabilities & Entfernen von einzelnen Systemrechten\\
			CGroups & Verwalten der Systemressourcen\\
			\bottomrule
		\end{tabular}
	\caption{Von LXC genutze Kernel-Features}
	\label{tab:lxcKernel}
\end{table}
\todo{Was macht LXC heutzutage unpopulär? Weniger Komplex als VServer usw. Trotzdem komplex. Basis für viele Impl, dann replaced by runC}

\subsection{CF Warden: Innovation durch Vereinfachung}
\label{sec:geschichteCFWarden}
2011 wurde die erste kommerziellen Container-Runtime mit \gls{acr-cf} veröffentlicht. Diese basierte zu Beginn auf \gls{acr-lxc} und erweiterte diese mit eigenen Funktionen, wie einer REST API zur Steuerung und Verwaltung eines Container Clusters. Dabei setzt \gls{acr-cf} Warden auf eine Client-Server-Architektur.
\missingfigure{Abbildung Warden Stack (erst LXC dann eigene impl)}
\todo{Erklären wie Warden funktioniert, wass es gut macht und was nicht}

\subsection{LMCTFY}
\label{sec:geschichteLMCTFY}
\todo{Titel der Section LMCTFY: OpenSource und der Einfluß von Go ?}

\subsection{Docker: Ecosystem, Runtime und \gls{acr-saas} in einem}
\label{sec:geschichteDocker}
\todo{Docker über die zeit (erst LXC, dann libcotnainer, dann containerd und runC, Was macht Docker attraktiv, was machen andere "falsch")}

\section{Aktuelle Probleme und Lösungen: Orchestrierung, Sicherheit und Standards}
\label{sec:geschichte}
\todo{Orchestrierung durch K8, millionen von Container, ..., Sicherheit (dirty COW) usw.,}

\section{Zusammenfassung}
\label{sec:timeline}
\fref{tab:timelineContainers} gibt einen Rückblick auf die Geschichte der Container-Technologie. Dabei wird der zeitliche Hergang einzelner Funktionen und Runtimes in Bezug gestellt und aktuelle Themen aufgezeigt.
\begin{table}[h]
	\providecommand{\timeline}{\color{LightSteelBlue3}\makebox[0pt]{\textbullet}\hskip-0.5pt\vrule width 1pt\hspace{\labelsep}}
	\begin{center}
		\begin{tabular}{@{\,}r <{\hskip 3pt} !{\timeline} >{\raggedright\arraybackslash}p{9cm}}
			\toprule
			\multicolumn{2}{c}{\textbf{Benötigte Technologien}}		\\
			\midrule
			1979 & Unix V7 mit chroot								\\
			1998 & SELinux											\\
				 & AppArmor											\\
			1999 & Linux Capabilites								\\
			2000 & FreeBSD Jails									\\
			2001 & Linux VServer									\\
			2002 & Linux namespaces									\\
			2004 & Solaris Container								\\
			2005 & Open VZ											\\
			2006 & Google Process Container							\\
			2007 & Process Container in Linux Kernel als \glspl{acr-cgroup}\\
			2012 & Erste Stabile Version der Sprache Go				\\
			\midrule
			\multicolumn{2}{c}{\textbf{Container-Runtimes}}			\\
			\midrule
			2008 & \gls{acr-lxc}									\\
			2011 & \gls{acr-cf} Warden								\\
			2013 & \gls{acr-lmctfy}									\\
			     & \gls{acr-cf} Graden, Umstieg auf Go				\\
			2014 & \Gls{acr-appc} Spezifikation Release				\\
				 & rkt												\\
			2015 & LXD												\\
				 & runC												\\
			2016 & Windows Containers								\\
				 & \gls{acr-cf} Guardian Release, Support für runC	\\
			2017 & containerd v1.0.0								\\
				 & cri-o											\\
			\midrule
			\multicolumn{2}{c}{\textbf{Entwicklung Container-Ecosystem}}\\
			\midrule
			2011 & Initialer Release \gls{acr-cf}					\\
			2013 & Release Docker, erstes Container Ecosystem		\\
			2014 & Entwicklung \gls{acr-k8} startet					\\
			2015 & Gründung \gls{acr-cncf} und \gls{acr-oci}		\\
				 & Docker Swarm										\\
			2016 & "Dirty Cow" $\rightarrow$ Container Sicherheit	\\
				 & Apache Mesos v1.0.0 Release						\\
			2017 & Übernahme rkt und containerd in \gls{acr-cncf}	\\
				 & Release \gls{acr-oci} runtime-spec und image-spec\\
			\bottomrule
		\end{tabular}
		\caption{Timeline Container-Technologien \citep{ABriefHistoryofContainers:fromthe1970sto2017}}
		\label{tab:timelineContainers}
	\end{center}
\end{table}
