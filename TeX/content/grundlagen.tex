\chapter{Grundlagen}
\label{chap:grundlagen}
Dieses Kapitel behandelt alle Grundlagen, die für Linux Container benötigt werden. Dabei liegt der Schwerpunkt auf den bestehenden Standards, die Funktionsweise hinter Containern und der Vorgehensweise, um eigene isolierte Prozesse zu instanziieren.

Um ein besseres Verständnis für die Funktionsweise und benötigte Technologien zu geben, wird zudem behandelt, wie man eigene Prozesse, im Beispiel eine \gls{gls-eureka} Instanz, in einem Linux System vollständig unabhängig und voneinander isoliert ausführen kann und so die Separation erhält, die Container versprechen.

\section{Standards}
\label{sec:standards}

\subsection{Open Container Initiative}
\label{sec:oci}

\subsection{Cloud Native Computing Foundation}
\label{sec:cncf}

\section{Funktionsweise}
\label{sec:funktionsweise}

\section{Eigene Implementierung}
\label{sec:eigeneImpl}