\chapter{Grundlagen}
\label{chap:grundlagen}
Dieses Kapitel behandelt alle Grundlagen, die für Linux Container benötigt werden. Dabei liegt der Schwerpunkt auf den bestehenden Standards, die Funktionsweise hinter Containern und der Vorgehensweise, um eigene isolierte Prozesse zu instanziieren.

Um ein besseres Verständnis für die Funktionsweise und benötigte Technologien zu geben, wird zudem behandelt, wie man eigene Prozesse, im Beispiel eine \gls{gls-eureka} Instanz, in einem Linux System vollständig unabhängig und voneinander isoliert ausführen kann und so die Separation erhält, die Container versprechen.

\section{Standards}
\label{sec:standards}

\subsection{Open Container Initiative}
\label{sec:oci}
Die \gls{acr-oci} ist eine Initiative, die seit 2015 unter der Linux Foundation agiert. Das Ziel der \gls{acr-oci} ist es, einen offenen Standard für Container zu schaffen, sodass die Wahl der Container-Laufzeitumgebung nicht mehr zu Inkompatibilität führt. Dabei liegt der Fokus auf eine einfache, schlanke Implementierung \citep{OpenContainerInitiative}.

Die \gls{acr-oci}  arbeitet aktuell an zwei Spezifikationen. Die runtime-spec standardisiert die Laufzeitumgebung  von Containern. Dabei wird festgelegt, welche Konfiguration, Prinzipien und Schnittstellen Laufzeitumgebungen stellen müssen. Die runtime-spec wird mit der Beispielimplementierung der \gls{acr-oci} runC implementiert. Dabei ruft runC \gls{acr-os} Funktionen über die Binary libcontainer auf, die mittlerweile in die \gls{acr-oci} übergegangen ist. Docker und \gls{acr-cf} nutzen runC als Container-Laufzeitumgebung.

Das zweite Projekt der \gls{acr-oci} ist die image-spec. Dieses versucht einen Standard für \glspl{gls-image} zu definieren. Als Grundlage des Standards dient das \gls{gls-image} Format, welches von Docker, \gls{acr-rkt} und anderen Laufzeitumgebungen genutzt wird.
\todo{Image Std und Docker Images verstehen}

\subsection{Cloud Native Computing Foundation}
\label{sec:cncf}

\section{Funktionsweise}
\label{sec:funktionsweise}

\section{Eigene Implementierung}
\label{sec:eigeneImpl}